\documentclass{beamer}

\mode<presentation>
{
  \usetheme{Warsaw}
  \useoutertheme{infolines}
  \usecolortheme[RGB={125,173,51}]{structure}
  %\usetheme[height=7mm]{Rochester}
  % or ...

  \setbeamercovered{transparent}
  % or whatever (possibly just delete it)
}

\usepackage{multicol}
\usepackage{verbatim} 
\usepackage{listings}
\usepackage{tikz}
\usetikzlibrary{arrows}
\usetikzlibrary{shapes}
\tikzstyle{block}=[draw opacity=0.7,line width=1.4cm]

\lstloadlanguages{C++}
\lstnewenvironment{code}
	{%\lstset{	numbers=none, frame=lines, basicstyle=\small\ttfamily, }%
	 \csname lst@SetFirstLabel\endcsname}
	{\csname lst@SaveFirstLabel\endcsname}
\lstset{% general command to set parameter(s)
	language=C++, basicstyle=\footnotesize\ttfamily, keywordstyle=\slshape,
	emph=[1]{tipo,usa}, emphstyle={[1]\sffamily\bfseries},
	morekeywords={tint,forn,forsn},
	basewidth={0.47em,0.40em},
	columns=fixed, fontadjust, resetmargins, xrightmargin=5pt, xleftmargin=15pt,
	flexiblecolumns=false, tabsize=2, breaklines,	breakatwhitespace=false, extendedchars=true,
	numbers=left, numberstyle=\tiny, stepnumber=1, numbersep=9pt,
	frame=l, framesep=3pt,
}

\usepackage[spanish]{babel}
% or whatever

\usepackage[latin1]{inputenc}
% or whatever

\usepackage{times}
\usepackage[T1]{fontenc}
% Or whatever. Note that the encoding and the font should match. If T1
% does not look nice, try deleting the line with the fontenc.


\title[Estructuras] % (optional, use only with long paper titles)
{Estructuras para operadores asociativos en intervalos}

\author[Agustín Gutiérrez] % (optional, use only with lots of authors)
{~Agustín Santiago Gutiérrez}
% - Give the names in the same order as the appear in the paper.
% - Use the \inst{?} command only if the authors have different
%   affiliation.
\institute[UBA] % (optional, but mostly needed)
{
  %\inst{1}
  Facultad de Ciencias Exactas y Naturales\\
  Universidad de Buenos Aires
}
\date[TC Arg 2017] % (optional, should be abbreviation of conference name)
{Training Camp Argentina 2017}

% Acá se puede insertar el logo de la UBA
% \pgfdeclareimage[height=0.5cm]{university-logo}{university-logo-filename}
% \logo{\pgfuseimage{university-logo}}



% Delete this, if you do not want the table of contents to pop up at
% the beginning of each subsection:
\AtBeginSubsection[]
{
  \begin{frame}{Contenidos}
  \footnotesize
  \begin{multicols}{2} 
    \tableofcontents[currentsection, currentsubsection]
  \end{multicols}
  \end{frame}
}

\DeclareMathOperator*{\mimin}{min}
\DeclareMathOperator*{\mimax}{max}

% If you wish to uncover everything in a step-wise fashion, uncomment
% the following command: 

%\beamerdefaultoverlayspecification{<+->}


\begin{document}
\pgfdeclarelayer{background}
\pgfsetlayers{background,main}
\begin{frame}
  \titlepage
\end{frame}


\begin{frame}

``For the fashion of Minas Tirith was such that it was built on seven levels,
each delved into a hill, and about each was set a wall, and in each wall
was a gate.''

\hfill		J.R.R. Tolkien, ``The Return of the King''

\end{frame}

\begin{frame} 
\footnotesize
\frametitle{Contenidos} 
\begin{multicols}{2} 
\tableofcontents 
\end{multicols} 
\end{frame}

\section{Operaciones en rangos} % 15 Min

\subsection{Operadores asociativos}

\begin{frame}{¿Qué es un operador asociativo?}
  \begin{block}{Definición}
    Un operador binario $\triangleright$ se dice \textbf{asociativo}, si siempre se cumple que $(a \triangleright b) \triangleright c = a \triangleright (b \triangleright c)$
  \end{block}
  \pause
  Ejemplos:
  \begin{itemize}
  \item \textbf{Suma $+$ de enteros (o reales)}
  \item Producto $\cdot$ de enteros (o reales)
  \item \textbf{Máximo / Mínimo de enteros (o reales)}
  \item Operadores lógicos ``or'' y ``and'' (caso particular de Max y Min)
  \end{itemize}
  
\end{frame}

\begin{frame}{Más ejemplos}
  Notar que en general el operador podría no ser conmutativo:
  \begin{itemize}
  \item Producto de matrices
  \item Composición de funciones
  \item Concatenación de Strings
  \end{itemize}
  
\end{frame}

\begin{frame}{Elemento neutro}
  \begin{block}{Definición}
    Un \textbf{elemento neutro} para un operador $\triangleright$ es un $e$ con la propiedad de que para cualquier $x$ siempre se cumple $x \triangleright e = e \triangleright x = x$
  \end{block}
  \pause
  Ejemplos:
  \begin{itemize}
  \item $0$ para la suma
  \item $1$ para el producto
  \item $+\infty$ / $-\infty$ para Min / Max
  \item $True$ / $False$ para And / Or
  \item $I$ (matriz identidad) para el producto de matrices
  \item $id$ (función identidad) para la composición de funciones
  \item ``'' (string vacío) para la concatenación de Strings
  \end{itemize}
  
\end{frame}

\begin{frame}{Elemento neutro (cont)}
  Asumiremos por comodidad siempre que sea necesario que existe un elemento neutro.
  
  Notar que si no existiera uno para la operación con la que estemos trabajando, podemos siempre simular uno con un valor ``NULL'' y ``poniendo un if'',
  de forma tal que una operación de NULL con $x$ deja al $x$ intacto como resultado.
  
\end{frame}

\subsection{Queries importantes}

\begin{frame}{Problema de consulta de prefijos}
    \begin{block}{Problema:}
    Dado un arreglo $v$ de $n$ posiciones, numeradas de $0$ a $n-1$, y un operador asociativo $\triangleright$, es posible definir
     para $0 \leq i \leq n$, la \textbf{consulta} $P(i)$, cuyo resultado es $P(i) = v_0 \triangleright v_1 \triangleright \cdots \triangleright v_{i-1}$
  \end{block}
  Es decir, se consulta el resultado del operador $\triangleright$ sobre los primeros $i$ elementos.
  \pause
  
  \begin{itemize}
  \item Notar que cuando el rango es vacío, la respuesta es el elemento neutro (definición).
  \item Cuando el arreglo inicial nunca se modifica, se dice que el problema es \textbf{estático}.
  \item Si el arreglo se va modificando (generalmente, cambiando el valor en una cierta posición) con el paso del tiempo entre consultas, se dice que el problema es \textbf{dinámico}.
  \end{itemize}
  
\end{frame}

\begin{frame}{Problema de consulta de intervalos (rangos)}
    \begin{block}{Problema:}
    Dado un arreglo $v$ de $n$ posiciones, numeradas de $0$ a $n-1$, y un operador asociativo $\triangleright$, es posible definir
     para $0 \leq i \leq j \leq n$, la \textbf{consulta} $R(i,j)$, cuyo resultado es $R(i,j) = v_i \triangleright v_{i+1} \triangleright \cdots \triangleright v_{j-1}$
  \end{block}
  Es decir, se consulta el resultado del operador $\triangleright$ sobre el rango de elementos $[i,j)$.
  \pause
  
  \begin{itemize}
  \item Nuevamente cuando el rango es vacío, definimos la respuesta como el elemento neutro.
  \item Al igual que antes, el problema puede ser estático o dinámico según se realicen modificaciones en el arreglo.
  \end{itemize}
  
\end{frame}

\begin{frame}{Problema de consulta de intervalos (cont)}
  
  Una observación importante es que cuando el operador tiene inverso (Ejemplo: \textbf{la resta}, para el $+$), es posible resolver este problema
         utilizando estructuras que resuelvan consulta de prefijos, mediante:
         
         $$ R(i,j) = inv(P(i)) \triangleright P(j) \mbox{ (por ejemplo, } P(j) - P(i) \mbox{ para el operador +})$$
  
\end{frame}

\begin{frame}{Las estructuras de datos}
  
  \begin{itemize}
      \item Veremos estructuras que resuelven cada uno de los 4 problemas planteados, en todos los casos para un operador asociativo completamente arbitrario.
             
       $$\begin{array}{rcc}
                    & \mbox{Prefijos}       & \mbox{Intervalos} \\
           \mbox{Estático} & \mbox{\textbf{Tabla Aditiva}}  & \mbox{\textbf{Sparse Table}} \\
           \mbox{Dinámica} & \mbox{Fenwick Tree}   & \mbox{\textbf{Segment Tree}} \\
         \end{array}$$

      \item ``Orden de dificultad subjetivo'': \\ \textbf{TablaAditiva} < \textbf{SparseTable} < FenwickTree < \textbf{SegmentTree}
      
  \end{itemize}
  
\end{frame}

\begin{frame}{Las estructuras de datos (múltiples dimensiones)}
  
  \begin{itemize}
      \item Todas estas estructuras generalizan adecuadamente para resolver los correspondientes problemas en múltiples dimensiones.
      \item Por ejemplo para el caso 2D, podemos usar un Segment Tree donde cada hoja sea un nuevo Segment Tree.
      \item Incluso se podrían combinar de maneras diferentes, teniendo por ejemplo una tabla aditiva en donde cada celda guarda una Sparse Table.
      \item \textbf{Lo veremos solamente para el caso de Tablas Aditivas}, donde es más simple.
  \end{itemize}
  
\end{frame}

\section{Tablas aditivas} % 20 min
\subsection{Caso unidimensional}

\begin{frame}{Problema}
    \begin {itemize}
        \item Dado un arreglo $v$ de elementos, queremos encontrar una estructura eficiente que permita calcular
         $v_0 \triangleright v_1 \triangleright v_2 \triangleright \cdots \triangleright v_{i-1}$ para cualquier $i$.
        \item Supondremos desde ahora en toda esta parte que el operador es la suma: $\triangleright \equiv +$
        \item Todo lo que hagamos para cálculos de prefijos se traslada idéntico para cualquier otro operador asociativo.
        \item Además, para cualquier operador con inversos, podremos calcular intervalos de la misma manera que lo haremos.
    \end {itemize}
\end{frame}


\begin{frame}{Arreglo acumulado}
   \begin{block}{Definición}
      Dado un arreglo unidimensional $v$ de $n$ elementos $v_0, v_1, \cdots, v_{n-1}$, definimos el \textbf{arreglo acumulado} de $v$ como
      el arreglo unidimensional $V$ de $n+1$ elementos $V_0, \cdots, V_n$ y tal que:
      $$V_i = \sum_{j=0}^{i-1}{v_j}$$
  \end{block}
  %\pause
  Notar que $V$ es muy fácil de calcular en tiempo lineal utilizando programación dinámica (``un for sumando''):
  %\pause
  \begin{itemize}
    \item $V_0 = 0$
    %\pause
    \item $V_{i+1} = v_i + V_i, \forall \ 0 \leq i < n$
  \end{itemize}
\end{frame}

\begin{frame}{Respuesta de los queries}
  \begin{itemize}
      \item Una vez computada la tabla, las consultas $P(i)$ son directamente los valores $V_i$ calculados.
      \item La respuesta a un query $R(i,j)$ de intervalo $[i,j)$ se obtiene directamente como $ V_j - V_i$
      \item Luego podemos responder cada query en tiempo constante, computando una sola resta entre dos valores del arreglo acumulado.
  \end{itemize}
\end{frame}

\subsection{Caso bidimensional (sumar rectángulos)}

\begin{frame}{Arreglo acumulado}
   \begin{block}{Definición}
      Dado un arreglo bidimensional $v$ de $n \times m$ elementos $v_{i,j}$ con $0 \leq i < n, 0\leq j < m$, 
      definimos el \textbf{arreglo acumulado} de $v$ como el arreglo bidimensional $V$ de $(n+1) \times (m+1)$ elementos
      tal que:
      $$V_{i,j} = \sum_{a=0}^{i-1}\sum_{b=0}^{j-1}{v_{a,b}}$$
  \end{block}
  %\pause
  ¿Podremos calcular fácilmente $V$ en tiempo lineal como en el caso unidimensional?
  %\pause
  %\invisible<1-2>
  {
      \textbf{¡Sí!} Utilizando programación dinámica.
      %\pause
      \begin{itemize}
        \item $V_{0,j} = V_{i,0} = 0 \ \ \forall \ 0 \leq i \leq n, 0 \leq j \leq m$
        %\pause
        \item $V_{i+1,j+1} = v_{i,j} + V_{i,j+1} + V_{i+1,j} - V_{i,j} \ \ \forall \ 0 \leq i < n, 0 \leq j < m$
      \end{itemize}
  }
\end{frame}

\begin{frame}{Respuesta de los queries}
  \begin{itemize}
  \item Las queries de rectángulos generales a la tabla aditiva vendrán dadas por rectángulos $[i_1,i_2) \times [j_1,j_2)$ con $0 \leq i_1 \leq i_2 \leq n, 0 \leq j_1 \leq j_2 \leq m$.
  %\pause
  \item La respuesta al query $[i_1,i_2) \times [j_1,j_2)$ sera $Q(i_1,i_2,j_1,j_2) = \sum \limits _{a=i_1}^{i_2-1}\sum \limits _{b=j_1}^{j_2-1}{v_{a,b}}$
  %\pause
    \item Pero de manera similar a como hicimos para calcular el arreglo acumulado, resulta que:
    $$Q(i_1,i_2,j_1,j_2) = V_{i_2,j_2} - V_{i_1,j_2} - V_{i_2,j_1} + V_{i_1,j_1}$$
    %\pause
    \item Luego podemos responder cada query en tiempo constante computando sumas y restas de cuatro valores del arreglo acumulado.
  \end{itemize}
\end{frame}

\subsection{Caso general}

\begin{frame}{Arreglo acumulado}
    El caso general es completamente análogo... Dejamos las cuentas escritas rápidamente.
   \begin{block}{Definición}
      Dado un arreglo $v$ de $n_1 \times \cdots \times n_r$ elementos $v_{i_1,\cdots,i_r}$ con $0 \leq i_k < n_k, 1\leq k \leq r$, 
      definimos el \textbf{arreglo acumulado} de $v$ como el arreglo $V$ de $(n_1+1) \times \cdots \times (n_r+1)$ elementos
      tal que:
      $$V_{i_1,\cdots,i_r} = \sum_{k_1=0}^{i_1-1} \cdots \sum_{k_r=0}^{i_r-1}{v_{k_1,\cdots,k_r}}$$
  \end{block}
\end{frame}
\begin{frame}{Arreglo acumulado (Cálculo usando programación dinámica)}
   El cálculo de $V$ mediante programación dinámica viene dado por:
      \begin{itemize}
        \item $$V_{i_1, \cdots, i_r} = 0\ \ \ \mbox{si para algún $k$ resulta $i_k = 0$}$$
        %\pause
        \item $$V_{i_1+1,\cdots,i_r+1} = v_{i_1,\cdots,i_r} + \sum_{d_1=0}^1 \cdots \sum_{d_r=0}^1 {(-1)^{r+1 + \sum\limits_{k=1}^r{d_k}}V_{i_1+d_1,\cdots,i_r+d_r}}$$
                Donde en las sumatorias anidadas se excluye el caso en que $d_k = 1 \ \forall\ k$
      \end{itemize}
\end{frame}

\begin{frame}{Respuesta de los queries}
  \begin{itemize}
  \item Las queries a la tabla aditiva vendrán dadas por subarreglos $[{i_1}_0,{i_1}_1) \times \cdots \times [{i_r}_0,{i_r}_1)$
  %\pause
  \item La respuesta al query $[{i_1}_0,{i_1}_1) \times \cdots \times [{i_r}_0,{i_r}_1)$ será:
  %\pause
    \item
    $$Q({i_1}_0,{i_1}_1, \cdots ,{i_r}_0,{i_r}_1) = \sum_{d_1=0}^1 \cdots \sum_{d_r=0}^1 {(-1)^{r + \sum\limits_{k=1}^r{d_k}}V_{{i_1}_{d_1},\cdots,{i_r}_{d_r}}}$$
    %\pause
    \item Luego podemos responder cada query en tiempo constante computando sumas y restas de $2^r$ valores del arreglo acumulado.
  \end{itemize}
\end{frame}

\subsection{Aplicaciones prácticas}

\begin{frame}{Aplicaciones prácticas de las tablas aditivas}
  \begin{itemize}
     \item Se usan en procesamiento digital de imágenes:
        \begin{itemize}
            \footnotesize
            \item Crow, Franklin (1984). ``Summed-area tables for texture mapping'' SIGGRAPH '84: Proceedings of the 11th annual conference on Computer graphics and interactive techniques. pp. 207-212.
            \item Lewis, J.P. (1995). ``Fast template matching.'' Proc. Vision Interface. pp. 120-123.
            \item Viola, Paul; Jones, Michael (2002). ``Robust Real-time Object Detection''. International Journal of Computer Vision.
        \end{itemize}
     \item Sirven para manipular distribuciones de probabilidad multivariadas, por ejemplo $P(A_x \leq X \leq B_x \land A_y \leq Y \leq B_y)$.
     \item Podemos encontrarlas en inglés como ``Summed area table''
  \end{itemize}
\end{frame}

\subsection{Tarea}
\begin{frame}{Tarea}
  \begin{itemize}
     \item http://www.spoj.pl/problems/KPMATRIX/
     \item http://www.spoj.pl/problems/TEM/
     \item http://www.spoj.pl/problems/MATRIX/
  \end{itemize}
\end{frame}

\section{Sparse Table}

\subsection{Motivación}

\begin{frame}{Motivación (range minimum query)}
    \begin{itemize}
        \item Si no trabajamos con un operador con inverso, las tablas aditivas no sirven para calcular intervalos arbitrarios.
        \item Por ejemplo si queremos el mínimo de un intervalo en lugar de la suma, ya no podemos ``restar'' y hacer lo mismo.
        \item Idea: Guardar prefijos es poco. Guardemos el acumulado de más intervalos del arreglo.
        \item Guardaremos el acumulado de todos los intervalos del arreglo con \textbf{tamaño potencia de 2}
        \item Notación: Usaremos en esta sección el operador Min, y llamaremos $RMQ(i,j)$ a la consulta de rango $[i,j)$
    \end{itemize}
\end{frame}

\subsection{Estructura}

\begin{frame}[fragile]{Estructura : Inicialización}
\begin{itemize}
    \small
      \item Utilizaremos programación dinámica para llenar el arreglo $M$, definido como:
      $$M(i,k) = RMQ(i,i+2^k), 0 \leq i \leq n - 2^k$$
      %\pause
      \item Para esto basta notar:
        \begin{itemize}
          \item $M(i,0) = RMQ(i,i+1) = v_i$
          \item $M(i,k+1) = RMQ(i,i + 2^{k+1}) = \mimin(M(i,k),M(i+2^k,k))$
        \end{itemize}
      %\pause
      \item La tabla $M$ contiene $\Theta(n \lg n)$ elementos, luego la inicialización requiere $\Theta(n \lg n)$ tiempo y memoria.
    \end{itemize}
\begin{lstlisting}
    for (int i = 0; i < N; i++) M[i][0] = v[i];
    for (int k = 0; k < ULTIMO_NIVEL; k++) {
        const int LARGO = (1<<k);
        for (int i = 0; i <= N - 2 * LARGO; i++)
            M[i][k+1] = min(M[i][k], M[i + LARGO][k]);
    }
\end{lstlisting}
\end{frame}

\begin{frame}{Estructura : Queries en $O(\lg N)$ (``estilo skiplist'')}
\begin{itemize}
      \item Una vez computado el arreglo $M$, podemos consultar cualquier rango potencia de 2.
      \item Como en binario todo número $N$ se escribe con $O(\lg N)$ potencias de 2, basta usar $O(\lg L)$ queries para formar un intervalo de longitud $L$.
      \item Por ejemplo el intervalo $[2,13)$, tiene longitud $11 = 8 + 2 + 1$
      \item Luego podemos formar $RMQ(2,13) = min(M(2,3) + M(10, 1) + M(12, 0))$
      \item Esto funciona con \textbf{cualquier operador asociativo}
    \end{itemize}
\end{frame}


\begin{frame}{Estructura : Queries en $O(1)$}
\begin{itemize}
      \item Una vez computado el arreglo $M$, para responder un query basta notar que:
      $$RMQ(i,j) = \mimin(M(i,k), M(j-2^k, k)), 2^k \leq j-i < 2^{k+1}$$
      %\pause
      \item Luego podemos responder queries en $O(1)$, siempre y cuando podamos computar $k$ en $O(1)$.
      %\pause
      %\invisible<1-2>
      {
        \item Esto se puede lograr al trabajar con enteros de 32 bits haciendo:
        \item \texttt{k = 31 - \_\_builtin\_clz(j-i)} en C++,
        \item \texttt{k = 31 - Integer.numberOfLeadingZeros(j-i)} en Java.
        %\pause
        \item Notar que esto \textbf{no funciona} con cualquier operador asociativo: se debe cumplir $x \triangleright x = x$ para todo $x$
      }
    \end{itemize}
\end{frame}

\subsection{Lowest Common Ancestor} % 20 min

\begin{frame}{El problema}
   \begin{block}{LCA}
      Dado un árbol con raíz de $n$ nodos, una estructura de LCA permite responder rápidamente consultas por el \textbf{ancestro común
      más bajo} entre dos nodos (es decir, el nodo más alejado de la raíz que es ancestro de ambos).
  \end{block}
  %\pause
  \begin{itemize}
      \item ¡Sorprendentemente se puede reducir una consulta de LCA a una de RMQ!
  \end{itemize}
\end{frame}

\begin{frame}{Recorrido de DFS}
\begin{itemize}
   \item Utilizando un recorrido de DFS, podemos computar un arreglo $v$ de $2n - 1$ posiciones que indica el orden en que
   fueron visitados los nodos por el DFS (cada nodo puede aparecer múltiples veces).
   %\pause
   \item Aprovechando este recorrido podemos computar también la \textbf{profundidad} (distancia a la raíz) de cada nodo $i$, que notaremos $P_i$.
   %\pause
   \item También guardaremos el índice de la primera aparición de cada nodo $i$ en $v$, que notaremos $L_i$ (Cualquier posición servirá, así
   que es razonable tomar la primera).
\end{itemize}
\end{frame}

\begin{frame}{Utilizacion del RMQ}
\begin{itemize}
   \item La observación clave consiste en notar que para dos nodos $i,j$ con $i \neq j$:
   $$ LCA(i,j) = RMQ(\mimin(L_i,L_j),\mimax(L_i,L_j)+1) $$
   \item Tomamos mínimo y máximo simplemente para asegurar que en la llamada a $RMQ$ se especifica un rango válido ($i < j$).
   \item Notar que en la igualdad anterior, $RMQ(i,j)$ compara los elementos de $v$ por sus valores de profundidad dados por $P$.
   %\pause
   \item La complejidad de la transformación del LCA al RMQ es $O(n)$, con lo cual la complejidad final de las operaciones
   será la del $RMQ$ utilizado.
\end{itemize}
\end{frame}

\subsection{Aplicaciones prácticas} % 

\begin{frame}{Aplicaciones prácticas}
\begin{itemize}
   \item Mediante LCA con Sparse Table, es posible computar en $O(N \lg N + Q)$, $Q$ consultas de distancia entre nodos en un árbol.
   \item Sparse Table puede combinarse también con estructuras de Strings (Suffix Array) para responder queries de ``prefijo común más largo'' entre
           subcadenas arbitrarias de cadenas dadas. Esto se puede usar en bioinformática para hacer comparaciones de ADN y buscar genes.
\end{itemize}
\end{frame}

\section{Segment Tree}

\subsection{Segment Tree}

\begin{frame}{Segment Tree}
    \begin{itemize}
      \item En esta sección seguiremos llamando ``mínimo'' a la operación $\triangleright$
      \item Segment Tree funciona bien con cualquier operador asociativo, en todos los casos
    \end{itemize}
\end{frame}

\begin{frame}{Estructura : Descripción}
    \begin{itemize}
      \item Para trabajar con el segment tree asumiremos que $n$ es potencia de $2$. De no serlo, basta extender el arreglo $v$ con a lo sumo $n$
            elementos adicionales para que sea potencia de $2$.
      %\pause
      \item Utilizaremos un árbol binario completo codificado en un arreglo: 
      %\pause
      \item $A_1$ guardará la raíz del árbol.
      \item Para cada $i$, los dos hijos del elemento $A_i$ serán $A_{2i}$ y $A_{2i+1}$. 
      \item El padre de un elemento $i$ será \texttt{i/2}, salvo en el caso de la raíz.
      \item Las hojas tendran índices en $[n,2n)$.
      %\pause
      \item La idea será que cada nodo interno del árbol (guardado en un elemento del arreglo) almacene el mínimo entre sus dos hijos.
      %\pause
      \item Las hojas contendrán en todo momento los elementos del arreglo $v$.
    \end{itemize}
\end{frame}

\begin{frame}{Estructura : Inicializacion}
    \begin{itemize}
       \item Llenamos $A_n, \cdots, A_{2n-1}$ con los elementos del arreglo $v$.
       %\pause
       \item Usando programación dinámica hacemos simplemente:
        $$A_i = \mimin(A_{2i},A_{2i+1})$$
        \item Notar que se debe recorrer $i$ en forma descendente para no utilizar valores aún no calculados.
        %\pause
        \item El proceso de inicialización toma tiempo $O(n)$, y la estructura utiliza $O(n)$ memoria.
    \end{itemize}
\end{frame}

\begin{frame}{Estructura : Modificaciones}
    \begin{itemize}
       \item Para modificar el arreglo, utilizaremos nuevamente la fórmula:
       $$A_i = \mimin(A_{2i},A_{2i+1})$$
       %\pause
       \item Notemos que si cambiamos el valor de $v_i$ por $x$, debemos modificar $A_{n+i}$ haciéndolo valer $x$, y recalcular
          los nodos internos del árbol...
       %\pause
       \item Pero sólo se ven afectados los ancestros de $A_{n+i}$.
       %\pause
       \item Luego es posible modificar un valor de $v$ en tiempo $O(\lg n)$, recalculando sucesivamente los padres desde $A_{n+i}$ hasta llegar a la raíz.
    \end{itemize}
\end{frame}


\begin{frame}{Estructura : Queries}
     Consideremos el siguiente algoritmo recursivo:
    \begin{itemize}
         \item $$f(k,l, r, i,j) = \left\{ \begin{array}{ll}
                                     A_k & \mbox{si $i \leq l < r \leq j$};\\
                                     +\infty & 
                                      \mbox{si $r \leq i$ o $l \geq j$} \\
                                     \mimin(f(2k,l,\frac{l+r}{2},i,j), & \\
                                     ,f(2k+1,\frac{l+r}{2},r,i,j))& sino.\end{array} \right.$$
         \item La respuesta vendrá dada por $RMQ(i,j) = f(1,0,n,i,j)$
         %\pause
         \item La complejidad temporal de un query es $O(\lg n)$
    \end{itemize}
\end{frame}

\subsection{Ejemplo}

\begin{frame}{Ejemplo}
    ``Suma máxima de dos elementos en rango''
    \begin{itemize}
         \item http://www.spoj.com/problems/KGSS/
    \end{itemize}
\end{frame}



\subsection{Aplicaciones prácticas}

\begin{frame}{Aplicaciones prácticas}
    \begin{itemize}
         \item Es usual que los algoritmos de geometría computacional se vean beneficiados por el uso de Segment Trees. Por ejemplo,
                se utilizan los Segment Trees en el algoritmo para calcular ``área de la unión de $N$ rectángulos'' en $O(N \lg N)$
    \end{itemize}
\end{frame}

\section{Binary Indexed Tree} % 20 min

\subsection{Binary Indexed Tree} % 20 min

\begin{frame}{Binary Indexed Tree}
  \begin{itemize}
    \item También llamados ``Fenwick Tree''
    \item Introducidos en la literatura en: \\
            Peter M. Fenwick (1994). ``A new data structure for cumulative frequency tables''. Software: Practice and Experience
    \item Se utilizan para implementar eficientemente algoritmos de compresión adaptativa mediante Arithmetic Encoding (motivación original de Fenwick)
    \item $\frac{\mbox{Fenwick}}{\mbox{Tabla Aditiva}} = \frac{\mbox{Segment Tree}}{\mbox{Sparse Table}}$ 
    \item Un Segment Tree es estrictamente más poderoso como estructura de datos. El Fenwick Tree tiene la misma complejidad, pero suele ser más fácil
           de programar y tener mejores constantes.
  \end{itemize}
\end{frame}

\end{document}
